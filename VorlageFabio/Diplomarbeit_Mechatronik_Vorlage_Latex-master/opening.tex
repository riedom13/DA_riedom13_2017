\frontmatter												%Seitennumerierung
\pagenumbering{Roman}										%Römische Zahlen
\addtocounter{page}{2}

\newcommand{\doublesignature}[2]{%
  \parbox{\textwidth}{
    \hfill
    \parbox{7cm}{
      \centering
      \rule{6cm}{1pt}\\
      #1
    }
    \parbox{7cm}{
      \centering
      \rule{6cm}{1pt}\\
      #2
    }
  }
  \mbox{}\\
  \mbox{}\\
  \mbox{}\\
  \mbox{}\\
}

\vspace*{20pt}

\section*{Eidestattliche Erklärung}
\label{sec:eidestattliche-erklaerung}
Ich erkläre an Eides statt, dass ich die vorliegende Arbeit selbstständig verfasst, andere als die angegebenen
Quellen/Hilfsmittel nicht benutzt und die den benutzten Quellen wörtlich und inhaltlich entnommenen
Stellen als solche kenntlich gemacht habe.\\
\\
Arnfels, am 05. April 2018\\

\vskip 2cm

\doublesignature{Lukas Freyler}{Stefan Ornik}
\doublesignature{Fabio Pölzl}{Dominik Riegelnegg}

\vskip 5cm

\clearpage

\newpage
\thispagestyle{empty}
\mbox{}

\clearpage

\section*{Danksagung}
\label{sec:danksagung}
An dieser Stelle möchten wir uns bei allen bedanken, die uns im Rahmen der Diplomarbeit unterstützt und betreut haben.\\
\\
Besonders bedanken möchten wir uns bei unseren Betreuern, Dipl.-Ing. Werner Harnisch, Dipl.-Ing. Wolfgang Mader, BEd. Otto Schuller
und Dipl.-Ing. Manfred Steiner für die fachliche Unterstützung während dieser Arbeit.

\newpage

\section*{Zusammenfassung}

Das Hauptziel der Diplomarbeit ist es, Pferden mehr Bewegung zu ermöglichen. In der freien Natur legen Pferde einige Kilometer am Tag zurück. Dies können die Pferde in den verschiedensten Reitställen nicht behaupten. Die Haltung eines Pferdes bringt viele Kosten mit sich. Da ist meistens nichts mehr für ein neues Trainingsgerät für die Pferde übrig. Als nächstes kommt hinzu, dass manche Personen auch nicht die nötige Zeit haben, um neben der Pferdeführanlage zu stehen und den Betrieb zu überwachen. Nun ergeben sich folgende Punkte, um die Schwierigkeiten für den Besitzer zu minimieren: \\
Unser Ziel ist eine automatisierte Anlage zu entwickeln, welche einfach über eine Android Applikation zu steuern ist. Da die Anlage auch an heißen Tagen betrieben wird, wurde eine Kühlung für die Pferde auf die Liste mit den Abarbeitungspunkten gesetzt. Die Erfrischung soll mittels einem Sprühregen realisiert werden, damit die Pferde bei heißen Temperaturen eine Abkühlung bekommen können. Der Benutzer muss ein bis vier Pferd/e innerhalb der Anlage positionieren und die Anlage starten. Die Zeit und die Geschwindigkeit der Pferdeführanlage soll er selbst auswählen können. Als letzten Punkt soll es eine Videoübertragung zur Android Applikation geben, damit der Benutzer jederzeit einen Sicherheitsblick auf die Anlage werfen kann, ob alles noch in Ordnung ist.  \newline{}

Was sind nun die wichtigsten Punkte von unserer Anlage?\\

\begin{itemize}
\item{Automatisierte Anlage}
\item{Platz für vier Pferde gleichzeitig}
\item{Android Applikation um die Anlage zu steuern}
\item{Videoüberwachung, die in der App abrufbar ist}
\item{Einfache Bedienung}
\item{Geschwindigkeit und Zeit sollen einstellbar sein} 
\item{Sprühregen zur Abkühlung der Pferde}
\item{Langlebig}
\end{itemize}

\newpage

\section*{Abstract}
\label{sec:abstractIntroduction}

The opportunity to kick up the scope of horses is the primary goal of this thesis. Horses are traveling many kilometers a day in the case in living in the wildlife. Horses which are the most time of their life inside the stable are not able to claim this. The keeping of animals also brings many costs with it. For that reason there isn't any money left for a new training device for their horses. The next point of difficulties is, that the keeper of the animals do not have the necessitative time to stand by the horse exerciser and control the system. To rationalize some of the problems the following issueses are need to be pursued: \\
The aim of the thesis, is to develop an automated system which could be controlled by an Android application easily. Because of the reason, that the horse exerciser is also going to operate during hot days, one point on the processing list is to make a cooling for the horses happen. If the user want to train some horses, he has to place up to four horses inside the horse exerciser. Now he is able to start the system. He should also be able to set the operating time and speed of the horse exerciser. The last point is, because of the reason of an video control, the user could throw a safety view on the system, that everything is alright.

What are the primary goals for the horse exerciser?

\begin{itemize}
	\item{automated unit}
	\item{capacity for four horses simultaneously}
	\item{Android application to control the system}
	\item{video control which is available inside the app}
	\item{easy handling}
	\item{light drizzle to cool down the horses}
	\item{durable}
\end{itemize} 


\clearpage

\newpage
\thispagestyle{empty}
\mbox{}

\clearpage

\subsection*{Gender Erklärung}
\label{sec:gender-erklaerung}
Aus Gründen der besseren Lesbarkeit wird in dieser Arbeit die Sprachform des generischen Maskulinums angewendet. Es wird an dieser Stelle darauf hingewiesen, dass die ausschließliche Verwendung der männlichen Form geschlechtsunabhängig verstanden werden soll.

\subsection*{Über dieses Dokument}
\label{sec:ueber-dokument}
Diese Arbeit wurde in \LaTeX{} verfasst. Diese Art der Dokumentation bietet gegenüber den normalen Textverarbeitungen gewisse Vorteile hinsichtlich der Formatierung und des Einbindens von Grafiken. Auch Formeln können sehr einfach und effizient angegeben werden.

\clearpage

\newpage
\thispagestyle{empty}
\mbox{}

\clearpage

\section*{Projektteam}
\label{sec:projektteam}

\subsection*{Lukas Freyler}
\begin{wrapfigure}[10]{l}{0.5\textwidth}
\begin{center}
  \includegraphics[width=0.35\textwidth]{fig/logoMecha}
\end{center}
\end{wrapfigure}
\mbox{}\\
\mbox{}\\
\textbf{Aufgabenbereich}:\\
\LaTeX{}\\
\textbf{Betreuer}:\\
Dipl.-Ing. Manfred Steiner
\mbox{}\\
\mbox{}\\
\mbox{}\\
\mbox{}\\
\mbox{}\\
\mbox{}\\

\subsection*{Stefan Ornik}
\begin{wrapfigure}[10]{l}{0.5\textwidth}
\begin{center}
  \includegraphics[width=0.35\textwidth]{fig/logoMecha}
\end{center}
\end{wrapfigure}
\mbox{}\\
\mbox{}\\
\textbf{Aufgabenbereich}:\\
\LaTeX{}\\
\textbf{Betreuer}:\\
Dipl.-Ing. Wolfgang Mader
\mbox{}\\
\mbox{}\\
\mbox{}\\
\mbox{}\\
\mbox{}\\
\newpage

\subsection*{Fabio Pölzl}
\begin{wrapfigure}[10]{l}{0.5\textwidth}
\begin{center}
  \includegraphics[width=0.35\textwidth]{fig/logoMecha}
\end{center}
\end{wrapfigure}
\mbox{}\\
\mbox{}\\
\textbf{Aufgabenbereich}:\\
\LaTeX{}\\
\textbf{Betreuer}:\\
Dipl.-Ing. Werner Harnisch
\mbox{}\\
\mbox{}\\
\mbox{}\\
\mbox{}\\
\mbox{}\\
\mbox{}\\

\subsection*{Dominik Riegelnegg}
\begin{wrapfigure}[10]{l}{0.5\textwidth}
\begin{center}
  \includegraphics[width=0.35\textwidth]{fig/logoMecha}
\end{center}
\end{wrapfigure}
\mbox{}\\
\mbox{}\\
\textbf{Aufgabenbereich}:\\
\LaTeX{}\\
\textbf{Betreuer}:\\
BEd. Otto Schuller
\mbox{}\\
\mbox{}\\
\mbox{}\\
\mbox{}\\
\mbox{}\\
\newpage